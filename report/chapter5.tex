\chapter{Network}

\section{Round trip time}
In this part, we try to calculate the round trip time.

\paragraph{Methodology}
The method is very simple, it's just like an echo server. Client send one byte to server, when server received the message then send back the message, and client receive it. Before client send message, we record current cycle; After client receive message, we record again. We iterate this procedure for 1000 times and get the average.

When iterating 1000 times and use the same opened connection. We did not use different connection, because it will add many TCP ACK overhead for connection. This experiment is only for RTT. The connection overhead will be measured in the following experiments.

\paragraph{Predictions}
Our predictions exclude the factor of building connection. So for loopback and remote interface, the biggest difference is send and receive data.

For loopback interface. We measure the hardware cycles are 20000, software cycles are 5000.

For remote interface. We measure the hardware cycles are 100000, software cycles are 10000.

\paragraph{Results}
We present our measured results.

\begin{center}
\begin{tabular}{l*{6}{c}r}
Operation       &  Hardware & Software & Overall & Measured\\
\hline
loopback & 20000 & 5000 & 25000 & 69882 \\
remote & 100000 & 10000& 110000 & 627772 \\
\end{tabular}
\end{center}

Following is the Ping results.
\begin{center}
\begin{tabular}{l*{6}{c}r}
Operation       &  RTT & Corresponding Cycles\\
\hline
loopback & 0.070ms & 161000 \\
remote & 0.196ms &  450800 \\
\end{tabular}
\end{center}

\paragraph{Discussion}
The overhead of network communication is exceeding our predictions. Even the loopback interface is over 60000 cycles for only 1 byte transfer. \\


What can you deduce about baseline network performance and the overhead of OS software?  \\
In this experiment we only transfer 1 byte data. In the next experiment, we will go deep into the network performance and the overhead of OS software. Because bandwidth benchmark will reveal more about this.

How close to ideal hardware performance do you achieve? \\ 
It is far short of the ideal hardware performance because of the TCP/IP protocol.\\


What are reasons why the TCP performance does not match ideal hardware performance?  \\
It is expensive to establish TCP reliable connection, much time on handshake, including ACK packages and SYN packages.

\section{Peak bandwidth}
In this part, we try to explore peak bandwidth.

\paragraph{Methodology}
The code is very similar to the previous experiment.

In this experiment, when we establish the connection, the server will send some data to the client and the client will receive this chunk of data. The size of data is a parameter. We use several sizes like 1000000 bytes, the result is pretty similar.

Besides, we repeat our procedure more than 10 times and choose the maximum result as peak bandwidth.

\paragraph{Predictions}
According to the memory bandwidth provided by hardware information, we think the peak bandwidth of loopback interface is 1600MB/s. 

Because the network bandwidth is 1000Mb/s, which is about 125 MB/s. We predict the remote interface is 125MB/s.

\paragraph{Results}
We present our measured results.

\begin{center}
\begin{tabular}{l*{6}{c}r}
Operation       &  Predicted & Measured\\
\hline
loopback for peak bandwidth & 1600 MB/s & 1540 MB/s \\
remote for peak bandwidth & 125 MB/s  & 133 MB/s\\
\end{tabular}
\end{center}


\paragraph{Discussion}
Our predictions are very close to the measured results. For loopback interface, the speed is close to the memory bandwidth; For remote interface, the speed is close to the network bandwidth, we think it is because these two machines are in the same LAN.

We also think the bandwidth can be influenced by many factors. For example, if the two machine are far from each other, then the route path is undecidable and the bandwidth will be influenced. \\

How close to ideal hardware performance do you achieve? \\ 
It is close to the hardware performance due to the perfect experiment environment.